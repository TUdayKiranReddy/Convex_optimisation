\documentclass{article}
\usepackage[utf8]{inputenc}
\usepackage{tabularx} % extra features for tabular environment
\usepackage{amsmath}  % improve math presentation
\usepackage{graphicx} % takes care of graphic including machinery
\usepackage{xspace}
\usepackage{tikz}
\usepackage{enumitem}
\usetikzlibrary{babel}
\usepackage[american]{circuitikz}
\usetikzlibrary{calc}
\usepackage{float}
\usepackage{siunitx}
\usepackage{pgfplots}
\usepackage{amsfonts} 
\usetikzlibrary{intersections}
\usepgfplotslibrary{fillbetween}
\usepackage[skins,theorems]{tcolorbox}
\tcbset{highlight math style={enhanced,
  colframe=red,colback=white,arc=0pt,boxrule=1pt}}
\pgfplotsset{width=10cm,compat=1.9}
\usepackage[margin=1in,letterpaper]{geometry} % decreases margins
\usepackage{cite} % takes care of citations
\usepackage[final]{hyperref} % adds hyper links inside the generated PDF file
\hypersetup{
colorlinks=true,       % false: boxed links; true: colored links
linkcolor=blue,        % color of internal links
citecolor=blue,        % color of links to bibliography
filecolor=magenta,     % color of file links
urlcolor=blue        
}

\begin{document}

\title{\textbf{CONVEX OPTIMISATION}\\{\textbf{TUTORIAL 12
}}}
\author{\textbf{TADIPATRI UDAY KIRAN REDDY}\\\textbf{EE19BTECH11038}}
\maketitle

\section*{(a)}
Given $f(x) = sup_{c \in \mathcal{C}}\overline{c}^T\overline{x}$. Clearly $\overline{c}^T\overline{x}$ is an affine function and it is convex. Operation $sup$ will preserve the convexity of the function. Thus $f$ is convex.
\section*{(b)}
Given primal (P),
\begin{equation*}
	\begin{aligned}
		\max \quad & \overline{c}^T\overline{x}\\
		\textrm{subject to} \quad & \mathbf{F}\overline{c} \le \overline{g}
	\end{aligned}
	\equiv 
	\begin{aligned}
		\min \quad & -\overline{c}^T\overline{x}\\
		\textrm{subject to} \quad & \mathbf{F}\overline{c} \le \overline{g}
	\end{aligned}
	\equiv f(\overline{x})
\end{equation*}
Now we find the dual of this problem by taking infimum of Lagrange function over $\overline{x}$.
\begin{gather*}
	\mathcal{L}(\overline{x}, \overline{\lambda}) =  -\overline{c}^T\overline{x} + \lambda ^T\left(\mathbf{F}\overline{c} - \overline{g}\right)\\
	g(\overline{\lambda}) = inf_{\overline{x}}\mathcal{L}(\overline{x}, \overline{\lambda})\\
\end{gather*}
Therefore the dual problem (D) is,
\begin{equation*}
	\begin{aligned}
		\min \quad & \overline{\lambda}^T\overline{g}\\
		\textrm{subject to} \quad & \mathbf{F}^T\overline{\lambda} = \overline{x}\\
		& \lambda \ge 0
	\end{aligned}
\end{equation*}
$f(\overline{x})$ is the optimal of (P) and dual gap for LPs is zero. Thus $f(\overline{x})$ is the optimal for (D) also.
\section*{(c)}
We can replace the objective in the \textit{Robust LP} with the dual problem.
\begin{equation*}
	\begin{aligned}
		\min _{\overline{x}} \min _{\overline{\lambda}} \quad & \overline{\lambda}^T\overline{g}\\
		\textrm{subject to} \quad & \mathbf{A}\overline{x} \ge \overline{b}\\
		& \mathbf{F}^T\overline{\lambda} = \overline{x}\\
		& \lambda \ge 0
	\end{aligned}
\end{equation*}
Clearly both $\overline{x}$ and $\overline{\lambda}$ thus the above problem can might as well written as,
 \begin{equation*}
	\begin{aligned}
		\min _{\overline{x}, \overline{\lambda}} \quad & \overline{\lambda}^T\overline{g}\\
		\textrm{subject to} \quad & \mathbf{A}\overline{x} \ge \overline{b}\\
		& \mathbf{F}^T\overline{\lambda} = \overline{x}\\
		& \lambda \ge 0
	\end{aligned}
\end{equation*}
Initial problem was not convex because both $\overline{x}$ and $\overline{c}$ are variable and $\overline{c}^T\overline{x}$ is not convex. But once we tranform the subproblem to dual we get a LP.
\end{document}