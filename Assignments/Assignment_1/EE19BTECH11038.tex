\documentclass{article}
\usepackage[utf8]{inputenc}
\usepackage{tabularx} % extra features for tabular environment
\usepackage{amsmath}  % improve math presentation
\usepackage{graphicx} % takes care of graphic including machinery
\usepackage{xspace}
\usepackage{tikz}
\usepackage{enumitem}
\usetikzlibrary{babel}
\usepackage[american]{circuitikz}
\usetikzlibrary{calc}
\usepackage{float}
\usepackage{siunitx}
\usepackage{pgfplots}
\usepackage{amsfonts} 
\usepackage[skins,theorems]{tcolorbox}
\tcbset{highlight math style={enhanced,
  colframe=red,colback=white,arc=0pt,boxrule=1pt}}
\pgfplotsset{width=10cm,compat=1.9}
\usepackage[margin=1in,letterpaper]{geometry} % decreases margins
\usepackage{cite} % takes care of citations
\usepackage[final]{hyperref} % adds hyper links inside the generated PDF file
\hypersetup{
colorlinks=true,       % false: boxed links; true: colored links
linkcolor=blue,        % color of internal links
citecolor=blue,        % color of links to bibliography
filecolor=magenta,     % color of file links
urlcolor=blue        
}

\begin{document}

\title{\textbf{CONVEX OPTIMISATION}\\{\textbf{ASSIGNMENT 2}}}
\author{\textbf{TADIPATRI UDAY KIRAN REDDY}\\\textbf{EE19BTECH11038}}
\maketitle

\section*{\hfil Question 1}
The set \textit{C} can also be represented as,
\begin{equation*}
	C = \{\overline{x}: \mathbf{Y}^T\overline{x} \ge \overline{0}\}
\end{equation*}
Where $\mathbf{Y}$ is a matrix with columns as elements of set $S$.
\subsection*{A: It is not a Subspace.}
Choose $\overline{x_1}, \overline{x_2}$ from set \textit{C}, if this set is subspace then $\forall \alpha _1, \alpha _2 \in \mathbb{R}; \alpha _1\overline{x_1} + \alpha _2\overline{x_2}$ must also be in set \textit{C}.\\
Since $\alpha _1$ and $\alpha _2$ can also take negative value which can the inequatlity, this relation cannot be valid. Thus it is not a Subspace

\subsection*{B: It is not a Affine set.}
The same explanation holds, since the only difference between affine and subspace is that in case of affine $\alpha _1 + \alpha _2 = 1$, but they can take any real values thus can be negative as well. Thus in this case also the ineqality need not satisfy.


\subsection*{C: It is a convex set.}
Choose $\overline{x_1}, \overline{x_2}$ from set \textit{C},
\begin{gather*}
	\mathbf{Y}^T(\theta \overline{x_1}) \ge \overline{0}; \mathbf{Y}^T((1 - \theta )\overline{x_2}) \ge \overline{0}\\
	\implies \mathbf{Y}^T(\theta \overline{x_1} + (1 - \theta )\overline{x_2}) \ge \overline{0}
\end{gather*}
This means that for any $\theta \in [0, 1]$ the above inequality satisfies which means that the set is convex.

\subsection*{D: It is a cone.}
By definition of cone if any set is a cone then any positively scaled vector must also belong to that set.\\
\begin{gather*}
	\mathbf{Y}^T\overline{x} \ge \overline{0}\\
	\implies \mathbf{Y}^T(\theta \overline{x}) \ge \overline{0}; \text{Given } \theta \ge 0
\end{gather*}

Observe that above conditions satisfied despite the nature of matrix $\mathbf{Y}$. Thus we can say that \textbf{none of the above queries depend on structure of \textit{S}}.

\section*{\hfil Question 2}
Given $f_1(\overline{x}) = ||\overline{y}-\mathbf{A}\overline{x}||_2$ and $f_2(\overline(x)) = ||\overline{y}-\mathbf{A}\overline{x}||_2^2$,\\
\begin{list}{•}{\textbf{Useful properties}}
	\item Norm function is convex.
	\item Affine transformation of domain of the function does'nt change the convexity of function.
\end{list}

In case of $f_1(\overline{x})$, the domain is affine transformation of $\overline{x}$ and the norm is operated on it, from above properties the function is convex.\\
\begin{gather*}
	f_1(\overline{x}) = ||\Phi(\overline{x})||_2\\
	\Phi: \overline{x} \rightarrow \overline{y}-\mathbf{A}\overline{x}
\end{gather*}

In case of $f_2(\overline{x})$,
\begin{gather*}
	f_2(\overline{x}) = \overline{x}^T\mathbf{A}^T\mathbf{A}\overline{x} - 2\overline{y}^T\mathbf{A}\overline{x} + \overline{y}^T\overline{y}
\end{gather*}
The above function is quadratic. The function is convex iff $\mathbf{A}^T\mathbf{A}$ is positive semi definite.
\begin{gather*}
	\mathbf{A} = \mathbf{V}\mathbf{\Sigma}\mathbf{V}^T\\
	\implies \mathbf{A}^T\mathbf{A} = (\mathbf{V}\mathbf{\Sigma}\mathbf{V}^T)^T(\mathbf{V}\mathbf{\Sigma}\mathbf{V}^T) = \mathbf{V}\mathbf{\Sigma}^T\mathbf{\Sigma}\mathbf{V}^T
\end{gather*}
$\mathbf{\Sigma}^T\mathbf{\Sigma}$ has positive diagonal entries as above multiplication will yeild individual diagonal entries of $\mathbf{\Sigma}$ to square each other. Since eigen values of $\mathbf{A}^T\mathbf{A}$ are positive thus it is positive definite. Hence $f_2(\overline{x})$ is a convex function.
\section*{\hfil Question 3}
\subsection*{(a)}

\subsection*{(b)}
\section*{\hfil Question 4}
\section*{\hfil Question 5}
\section*{\hfil Question 6}
\end{document}